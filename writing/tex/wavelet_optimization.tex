\documentclass[../tech_report_1.tex]{subfiles}
\graphicspath{{img/}{../img/}}

\begin{document}

\section{Introduction to wavelet density estimation}

Wavelet density estimation was a promising technique for shape matching, as explored by Peter and Rangarajan in their 2008 paper\cite{peter2008maximum}. Generally, the goal is to estimate a probability density function over the shape points and use that representation to compare shapes. The density is estimated by expansion into a wavelet basis form, which has a theory with strong mathematical foundations and provides several advantages over competing function bases (e.g. the Fourier basis, or simple histograms). 

This wavelet density estimation framework can be applied for any dimensional data (e.g. 1D, 2D, and 3D shapes) by using the tensor product (multidimensional) forms of the 1D wavelet bases. The coefficients of this functional basis (of the probability density function) form our feature representation, and also conveniently map to the unit hypersphere due to constraints we impose during density estimation. This manifold geometry allows us to use a simple, efficient, and well-defined distance metric as our shape dissimilarity measure.

\section{Implementation and our optimization}

The wavelet density estimator code by Peter and Rangarajan accurately constructs the coefficient vectors and thus estimates the densities, but because of the sheer complexity of the procedure, there was significant room to improve the runtime. A full estimation over the standard shape databases can take multiple days even with relatively low resolution.
For instance, the Brown database\cite{sebastian2004recognition} has 99 shapes, the Swedish Leaf database\cite{soderkvist2001computer} has 1125 shapes, and the MPEG7 database\cite{jeannin1999description, bober2001mpeg} has 1400 shapes, which are quite large already; the Princeton ModelNet database is orders of magnitude larger, at 127,915 shapes. At these scales, the inefficiency of the Peter-Rangarajan code is obvious (i.e. ModelNet estimation becomes intractable). Optimization is thus a primary priority.

THe Peter-Rangarajan follows the full mathematical implementation, which means there exist superfluous calculations. For instance, they loop over every sample point and evaluate every basis function individually. We added an optimization of only calculating the basis functions
for relevant sample points under specified basis functions with the
hope to cut computation power (and thus, time required for estimation) considerably.

In this
report we first provide an exposition of  wavelets and its use in density estimation, delving deeper into the equations which drive the core of the implementation by Peter and Rangarajan. We then provide an explanation of the optimizations, and finally report the extremely promising results from said optimizations. 


\section{Wavelet Density Estimation}

Our feature representation is formed by estimation of the probability density function of the shape, which we model as sample points from a distribution. 

First, we must explain the use of wavelets as the foundation of our density estimation framework. An $L^{2}$ function, also known as a square-integrable function, is a real- or complex-valued function whose square integrates to a finite number. Probability density functions fall
under this function space, therefore allowing us to use wavelets for
density estimation using a linear combination of these basis functions. In other words, wavelet density estimation  as a paradigm is built on strong mathematical foundations.


The equation for 1D density estimation is 
\begin{equation}
p(x)=\underset{j_{o},k}{\sum}\alpha_{j_{o},k}\phi_{j_{o},k}(x)+\underset{j\geq j_{o},k}{\overset{j_{1}}{\sum}}\beta_{j,k}\psi_{j,k}(x)\label{eq:WDE}
\end{equation}
where $\phi(x)$ and $\psi(x)$ are the scaling and wavelet basis
functions, also known as the ``father'' and ``mother'' wavelets,
respectively. The functions can also be set to some starting resolution
level that determines its dilation and contraction. To extend to multiple
resolutions, we incorporate the wavelet basis function with a starting ($j_{o}$) and stopping ($j$) resolution level. In order to form a
basis, we need to make multiple copies of the function over different
translates, $k$. These functions are scaled by their scaling and
wavelet basis function coefficients, $\alpha_{j_{o},k}$ and $\beta_{j,k}$.
To easily retain properties of true densities, such as non-negativity and
integration to one, we compute the square root of the probability
density function 
\begin{equation}
\sqrt{p(x)}=\underset{j_{o},k}{\sum}\alpha_{j_{o},k}\phi_{j_{o},k}(x)+\underset{j\geq j_{o},k}{\overset{j_{1}}{\sum}}\beta_{j,k}\psi_{j,k}(x)\label{eq:sqrtWDE}
\end{equation}
Since we optimized for multi-resolution 2D wavelet density estimation,
we consider the second dimension with $\mathbf{x}=(x_{1},x_{2})\in\mathbb{R}^{2}$
and \textbf{$\mathbf{k}=(k_{1},k_{2})\in\mathbb{Z}^{2}$}. We also
extend the equation to 
\begin{equation}
\sqrt{p(\mathbf{x})}=\underset{j_{o},\mathbf{k}}{\sum}\alpha_{j_{o},\mathbf{k}}\phi_{j_{o},\mathbf{k}}(\mathbf{x})+\underset{j\geq j_{o},\mathbf{k}}{\overset{j_{1}}{\sum}}\overset{3}{\underset{w=1}{\sum}}\beta_{j,\mathbf{k}}^{w}\psi_{j,\mathbf{k}}^{w}(\mathbf{x})\label{eq:sqrtWDE2Dmultires}
\end{equation}
where we calculate for the basis functions by using the tensor product
method 
\[
\phi_{j_{o},\mathbf{k}}(\mathbf{x})=2^{j_{o}}\phi(2^{j_{o}}x_{1}-k_{1})\phi(2^{j_{o}}x_{2}-k_{2})
\]
\[
\psi_{j,\mathbf{k}}^{1}(\mathbf{x})=2^{j}\phi(2^{j}x_{1}-k_{1})\psi(2^{j}x_{2}-k_{2})
\]
\[
\psi_{j,\mathbf{k}}^{2}(\mathbf{x})=2^{j}\psi(2^{j}x_{1}-k_{1})\phi(2^{j}x_{2}-k_{2})
\]
\begin{equation}
\psi_{j,\mathbf{k}}^{3}(\mathbf{x})=2^{j}\psi(2^{j}x_{1}-k_{1})\psi(2^{j}x_{2}-k_{2})\label{eq:multiRes2D}
\end{equation}
A mathematical advantage in using wavelets is its compact support,
meaning they are nonzero on a small domain. Any point falling outside
of the basis function's support does not contribute to the basis function
value at that translation.

Intuitively, the scaling and wavelet basis
function coefficients determine the height of its basis function.
These coefficients thus uniquely define
the probability density function of a shape, which means we can use them as our feature representation
in shape retrieval. In order to calculate coefficients we use the
following equations:
\begin{equation}
\alpha_{j_{o},\mathbf{k}}=\bigg(\frac{1}{N}\bigg)\overset{N}{\underset{i=1}{\sum}}\frac{\phi_{j_{o},\mathbf{k}}(\mathbf{x})}{\sqrt{p(\mathbf{x})}}\label{eq:scalingCoeff}
\end{equation}
\begin{equation}
\beta_{j_{o},k}=\bigg(\frac{1}{N}\bigg)\underset{i=1}{\overset{N}{\sum}}\frac{\psi_{j,\mathbf{k}(\mathbf{x})}}{\sqrt{p(\mathbf{x})}}\label{eq:waveletCoeff}
\end{equation}

As we estimate the coefficients of our probability density function,
we use a loss function to determine whether our coefficients are as
close to optimal as possible. To do this we try to minimize a negative
loglikelihood objective function with respect to the coefficients
\begin{equation}
-\log p(X;\{\alpha_{j_{o},k}\beta_{j,k}\})=-\frac{1}{N}\underset{i=1}{\overset{N}{\sum}}\log\Bigg[\underset{j_{o},k}{\sum}\alpha_{j_{o},k}\phi_{j_{o},k}(x_{i})+\underset{j\geq j_{o},k}{\overset{j_{1}}{\sum}}\beta_{j,k}\psi_{j.k}(x_{i})\Bigg]^{2}\label{eq:costFunction}
\end{equation}
where $X=\{x_{i}\}_{i=1}^{N}$. To determine the direction
of the gradient descent, we compute the following expression:
\begin{equation}
-2\bigg(\frac{1}{N}\bigg)\sum\frac{\phi_{j_{o},\mathbf{k}}(\mathbf{x})}{\sum\alpha_{j_{o},\mathbf{k}}\phi_{j_{o},\mathbf{k}}(\mathbf{x})}\label{eq:gradient}
\end{equation}
In order for maintain essential properties of density estimation,
we constrain the coefficients to one
\begin{equation}
\underset{j_{o},k}{\sum}\alpha_{j_{o},k}^{2}+\underset{j\geq j_{o},k}{\overset{j_{1}}{\sum}}\beta_{j,k}^{2}=1\label{eq:constraint}
\end{equation}
This constraint conveniently maps our feature representation into
a unit hypersphere which allows us to take advantage of its geometric
properties to signify similarity and dissimilarity between shapes
in shape retrieval. 


\section*{Wavelet Density Estimation Optimization}

Now that we have an overview of wavelet density estimation and the
equations used, we can delve into our optimization. We focus our efforts on the 2D form of the density estimation on a given point set shape representation.
We then use the coefficients that uniquely characterize the density
function as our feature vector. 

Originally, for a database such as
MPEG7, which has 1400 shapes, an estimation process with a domain of {[}-0.5,0.5{]}, resolution
level 2 to 3, and 1344 translates would take roughly 16.8 hours.
This is a particularly low resolution, and if raised we would expect
an even slower runtime. To optimize this code for speed, we focus on the
bottleneck: the calculation of the initial coefficients and negative
log likelihood cost value with its gradient. 


\subsection*{Initializing Coefficients}

This function calculates equations (\ref{eq:scalingCoeff})
and (\ref{eq:waveletCoeff}) to compute initial coefficients
that will be updated later through a loss-minimizing optimization process.
The time spent for this calculation with the same parameters mentioned
above is roughly around 1.8 of the 16.8 hours. 

Given a point set shape representation, the wavelet density estimation framework covers the
shape with basis functions at each translation. Each function has
a coefficient that needs to be calculated. The bottleneck exists because of the way the code structures the computation: it
code computes equations (\ref{eq:multiRes2D}) for a \emph{single
point over all translations} using a Kronecker
tensor product. \cite{van2000ubiquitous}
 This means that if we have 1344 translations,
it would perform 1344 operations for a single point. If a shape is made up of 4007 sample
points, then it would be performing a total of 5,385,408 operations
for a single shape. This calls for an excess amount of redundant computations
since most of the scaling and wavelet basis functions for a single
point over all translations return the value of zero because the observed sample point does not contribute
to most basis functions, i.e. does not exist under the compact support
of the basis function at specific translations. The solution is to change the structure of the looping: 

Instead of looping through each point and finding which basis functions
it falls under, we looped through basis functions along the horizontal
direction, evaluated its basis function value (\ref{eq:multiRes2D})
based on the points fall under its support, and placed them into a
matrix that holds these values. We then store this matrix to be used
for later optimized computation (for the negative log likelihood cost
value and gradient). Once we have this matrix of values, we can evaluate the basis function coefficients with the relevant points
under each translation using equation (\ref{eq:scalingCoeff}) and
(\ref{eq:waveletCoeff}). 


\subsection*{Negative Log Likelihood}

The second area where the bottleneck occurs is in computing for the
negative log likelihood cost value (\ref{eq:costFunction}) and the
function's gradient (\ref{eq:gradient}). We use these equations to
check whether we have found the  optimal coefficient values and
which direction in which to move to optimize. The original time spent for this
computation was around 13.9 of the 16.8 hours. 

The original code actually replicated much of the work done in initialize coefficients. It would take a single point and calculate the basis
function over all translations, resulting in needless computation.
It does this to attain a matrix of the basis function values to perform
the appropriate operations to solve for the cost value and gradient. 

Since we already performed most of the heavy computation to attain
this matrix in initializing coefficients, our solution was to simply
pass in the needed matrix of basis function values for each translation
and perform the proper computations. This effectively solves for the
negative log likelihood cost value and gradient while ultimately eliminating
loops. 

\section*{Results}

As mentioned above, our goal is to estimate a density function of
a 2D shape using wavelets. This allows us to extract the coefficients
as our feature representation to be used in shape retrieval on a unit
hypersphere. However, with MPEG7 containing 1400 shapes, domain {[}-0.5,0.5{]},
resolution level 2 to 3, and 1344 translates, it would take about 16.8
hours to calculate the coefficients. Our optimization yields significant improvements over the original code.

The original time it took the code to calculate for only the initial
coefficients was about 1.8 hours. After our optimization, the run
time to compute the initial coefficients cut all the way down to 0.13
hours, or about eight minutes. Ultimately, the computation runs 92.8\%
faster. As for calculations for the negative log likelihood cost value
and gradient, our optimization shaved the lines of code from 54 lines
to 21 lines, ridding it of loops. We essentially do away with 61\%
of lines of code. The run time was promising as well. To calculate
the cost value and gradient for seven iterations, computation time
went from 13.9 hours to 0.089 hours, or around 5 minutes for the entire
dataset. The optimized computation runs 99\% faster. Overall, to estimate
the wavelet densities of a database of 2D shapes with the optimized
code under the specified parameters would take 0.3 hours, or 17.8
minutes, running 98\% faster.

With this optimization, we hope that researchers using the wavelet density framework can achieve full estimation of the coefficient values quickly, and spend more
time in developing new and better methods of shape retrieval.


\end{document}
