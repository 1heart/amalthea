\documentclass[../tech_report_1.tex]{subfiles}
\graphicspath{{img/}{../img/}}
\begin{document}


\section{Von Mises distribution}

Intuitively, the Von Mises distribution  \cite{wolfram_von_mises} is
a simple approximation for the normal distribution on a circle (known
as the \textit{wrapped normal distribution}). The Von Mises probability
density function is defined by the following equation:

\[
P(x)=\frac{{e^{b\cos(x-a)}}}{2\pi I_{0}(b)}
\]


where $I_{0}(x)$ is the modified Bessel function of the first kind;
the Von Mises cumulative density function has no closed form.

The mean $\mu=a$ (intuitively, the angle that the distribution clustered
around), and the circular variance $\sigma^{2}=1-\frac{{I_{1}(b)}}{I_{0}(b)}$
(intuitively, $b$ is the ``concentration'' parameter). Therefore,
as $b\rightarrow0$, the distribution becomes uniform; as $b\rightarrow\infty$,
the distribution becomes normal with $\sigma^{2}=1/b$.


\section{Von Mises-Fisher distribution}

The Von-Mises Fisher distribution is the generalization of the Von
Mises distribution to $n$-dimensional hyperspheres. It reduces to
the Von-Mises distribution with $n=2$. The probability density function
is defined by the following equation:

\[
f_{p}(\boldsymbol{x};\boldsymbol{\mu},\kappa)=C_{p}(\kappa)\exp(\kappa\boldsymbol{\mu}^{T}\boldsymbol{x})
\]


where

\[
C_{p}(\kappa)=\frac{\kappa^{p/2-1}}{(2\pi)^{p}I_{p/2-1}(\kappa)}
\]
 and intuitively is an approximation for the normal distribution on
the hypersphere.

\subsection{Implementation of random spherical data in MATLAB}

By using the Circular Statistics Toolbox in MATLAB, we can construct random spherical data through exploitation of the of Von Mises (circular) distribution.

\begin{verbatim}

%--------------------------------------------------------------------------
% Function:    random_spherical_data
% Description: Draws a line on a sphere.
% 
% Inputs: 
%
% numClusters     - Number of clusters.
%
% numPoints       - Number of points per cluster.

% numPoints       - Kappa (concentration parameter);
%                     higher means closer clusters.

% means1, means2  - Optional vector of means.
%
% 
% Outputs
% 
% dataMatrix      - An nxd dataMatrix containing the random points.
% meanMatrix      - A matrix containing the means of each of the clusters.
% memMatrix       - An array of labels for each randomly generated datapoint.
% 
%
% Usage: Used in hierarchical clustering on the unit hypersphere.
%
% Authors(s):
%   Mark Moyou - markmmoyou@gmail.com
% Yixin Lin - yixin1996@gmail.com
%   Glizela Taino - glizelataino@gmail.com
%
% Date: Monday 6th June, 2016 (2:34pm)
%
% Affiliation: Florida Institute of Technology. Information
%              Characterization and Exploitation Laboratory.
%              http://research2.fit.edu/ice/
% -------------------------------------------------------------------------



function [dataMatrix, meanMatrix, memMatrix] = random_spherical_data(...
  numClusters, numPoints, kappa, means1, means2)

if nargin <= 3
  means1 = nan(numClusters); means2 = nan(numClusters);
  for i = 1:numClusters
    means1(i) = rand_angle();
    means2(i) = rand_angle();
  end
end

if length(means1) ~= numClusters || length(means2) ~= numClusters
  error('Means vectors are not the right length!');
end

dataMatrix = []; meanMatrix = []; memMatrix = [];
for i = 1:numClusters
    mean1 = means1(i); mean2 = means2(i);
    % Create random angles using the von Mises distributions
    angles = [circ_vmrnd(mean1, kappa, numPoints) circ_vmrnd(mean2, kappa, numPoints)];
    % Convert to Cartesian coordinates
    [x ,y, z] = sph2cart(angles(:,1), angles(:, 2), 1);
    [muX, muY, muZ] = sph2cart(mean1, mean2, 1);

    meanMatrix = [meanMatrix; muX, muY, muZ];
    dataMatrix = [dataMatrix; [x y z]];
    memMatrix = [memMatrix; (i + zeros(numPoints, 1))];
end

end
\end{verbatim}

\end{document}
