\documentclass[../tech_report_1.tex]{subfiles}
\graphicspath{{img/}{../img/}}
\begin{document}

Shape analysis is the fundamental computer vision problem of helping machines understand shape. Here, we focus primarily on the subproblem of shape retrieval: given a \textit{query shape}, how can we best retrieve the most similar shapes (\textit{search results}) from an already-known catalog of shapes? Shape retrieval is as similarly fundamental to the problem of shape analysis as search engines to was to the Internet.

The standard general framework for shape retrieval is to consume some shape representation (e.g. 2D points, 3D point-clouds, or 3D meshes), extract a feature representation that captures the essence of the shape, and using retrieval mechanics to output a list of shapes based on similarity.

\section{Motivations}

There has been several developments which make shape analysis both exciting and crucial in recent years. First, the number of available models (2D and 3D shapes) available over the Internet, partly due to ever-decreasing costs of creating such models and partly due to the concerted efforts of researchers to construct domain-specific databases. Second, our understanding of the principles behind computer vision have advanced greatly, due to both the necessity of making sense of enormous amount of data being generated every day\cite{manyika2011big}, and meteoric advances in our ability to solve them\cite{krizhevsky2012imagenet}.

It is interesting to note that, fundamentally, our understanding of images and objects comes directly from our ability to understand their shape. To see this most intuitively, consider the two objects in figure \ref{fig:different_shapes} which we immediately recognize as opposite meanings, and two objects in figure \ref{fig:same_shapes} we recognize immediately as the same. Though the texture, color, scale, and amount of distortion varies tremendously in the two similar objects, shape enables us to tell them apart; on the other hand, all those attributes are constant between the two dissimilar objects but we can tell them apart extremely easily, again due to shape.


\begin{figure}[!ht]
  \centering
    \includegraphics[width=0.2\textwidth]{happy_face}
    \includegraphics[width=0.2\textwidth]{happy_face_2}
  \caption{Two happy faces. Notice that despite texture, color, sizing, and distortion changes, we can immediately tell them apart due to shape information. \label{fig:same_shapes}}
\end{figure}


\begin{figure}[!ht]
  \centering
    \includegraphics[width=0.2\textwidth]{happy_face}
    \includegraphics[width=0.2\textwidth]{unhappy_face}
  \caption{A happy and unhappy face. Notice that despite the same texture, color, sizing, and without any distortion, we can immediately tell them apart due to their shape. \label{fig:different_shapes}}
\end{figure}

\section{Applications}

Shape retrieval has a plethora of potential applications, which stems from how fundamental shape analysis is to the problem of computer vision. Solving shape retrieval has as far ranging effects as helping robots quickly make sense of their environment, helping build billion-dollar 3D animation franchises, and becoming an enabler of information retrieval for commerce and virtual/augmented reality on the scale that search engines did for text-based websites. Ultimately, fully understanding the implications of solving this problem may be impossible because the problem is so basic.

\end{document}
