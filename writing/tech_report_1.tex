\documentclass{article}

\usepackage{float}
\usepackage{amsmath}
\usepackage{amsbsy}
\usepackage{amsthm}
\usepackage{graphicx}
\usepackage{algorithm,algpseudocode}
\usepackage[english]{babel}
\usepackage{subfiles}
\usepackage{authblk}

\graphicspath{{img/}{../img/}}

\newtheorem*{theorem}{Theorem}

\begin{document}

\title{Hierarchical Shape Retrieval on the Unit Hypersphere}
\author{Glizela Taino}
\affil{Hawaii Pacific University, Honolulu, Hawaii}
\author{Yixin Lin}
\affil{Duke University, Durham, North Carolina}
\author{Mark Moyou}
\affil{Florida Institute of Technology, Melborne, Florida}
\author{Adrian M. Peter}
\affil{Florida Institute of Technology, Melborne, Florida}
\setcounter{Maxaffil}{0}
\renewcommand\Affilfont{\itshape\small}
\date{June 3, 2016}
\maketitle

\begin{abstract}
Shape retrieval is important because \_\_\_\_\_\_\_. Our goal is to
develop the most effective method of shape retrieval. We created the
feature representations of each shape in the data set with LBO signatures
and their probability density coefficients that map onto the surface
of a unit hypersphere. Then we recursively cluster the observations
to create a hierarchy based on centroid linkage. This essentially
builds a decision tree. When we are given a query, we preprocess it
in the same way, map it onto the manifold, and perform a model to
model comparison to decide which branches to traverse to get back
the most relevant results from the data base as effectively as possible.
Our results show \_\_\_\_\_\_\_\_. In conclusion \_\_\_\_\_\_\_\_.\end{abstract}


\subfile{tex/von_mises.tex}
\subfile{tex/k_means.tex}
\subfile{tex/spherical_k_means.tex}
\subfile{tex/hypersphere_means.tex}
\subfile{tex/hierarchical_clustering.tex}

\newpage
\bibliographystyle{plain}
\bibliography{tech_report_1.bib}

\end{document}
