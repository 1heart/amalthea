\documentclass{article}

\usepackage{float}
\usepackage{amsmath}
\usepackage{bm}
\usepackage{amsbsy}
\usepackage{amsthm}
\usepackage{amssymb}
\usepackage{graphicx}
\usepackage{algorithm,algpseudocode}
\usepackage[english]{babel}
\usepackage{subfiles}
\usepackage{authblk}
\usepackage[colorlinks=true]{hyperref}
\usepackage[a4paper, total={7in, 10in}]{geometry}
\usepackage{lipsum}
\usepackage{textcomp}

\graphicspath{{img/}{../img/}}

\newtheorem*{theorem}{Theorem}

\begin{document}
\date{July 2016}

\title{Hierarchical Shape Retrieval on the Unit Hypersphere}
\author{Glizela Taino}
\affil{Department of Mathematics, Hawaii Pacific University, Honolulu, Hawaii; \href{mailto:glizelataino@gmail.com}{glizelataino@gmail.com}}
\author{Yixin Lin}

\affil{Department of Computer Science and Mathematics, Duke University, Durham, North Carolina; \href{mailto:yixin.lin@duke.edu}{yixin.lin@duke.edu}}
\author{Mark Moyou}
\affil{Department of Engineering Systems, Florida Institute of Technology, Melbourne, Florida; \href{mailto:mmoyou@my.fit.edu}{mmoyou@my.fit.edu}}
\author{Adrian M. Peter}
\affil{Department of Engineering Systems, Florida Institute of Technology, Melbourne, Florida; \href{mailto:apeter@fit.edu}{apeter@fit.edu}}
\setcounter{Maxaffil}{0}
\renewcommand\Affilfont{\small}
\maketitle

\subfile{tex/abstract}
\subfile{tex/introduction}
The layout of this paper is as follows:
  % TODO: layout

\part{Shape classification and retrieval}
  \subfile{tex/intro_to_shape}
  \subfile{tex/shape_lit_review}
  \subfile{tex/laplace_beltrami}
  \subfile{tex/wavelet_density_estimation}

\part{Clustering on a sphere}

  % TODO: write intro for clustering on a sphere

  \subfile{tex/von_mises}
  \subfile{tex/k_means}
  \subfile{tex/spherical_k_means}
  \subfile{tex/hypersphere_means}

\part{Hierarchical clustering}

  % TODO: revise
  For our lab, we want to develop the most efficient way to group data
  based on a measure of dissimilarity. We explored the topic of
  hierarchical clustering which is used in data mining and statistics.
  It is visualized useing a dendogram to show the organization and relationship
  among clusters. Hierarchical clustering can be implemented in two
  ways: divisive and agglomerative. We will elaborate on these two variants
  in this paper to compare the its algorithmic complexity, or how much
  resources are needed when computed, and the different types of cluster
  linkage criterion. Further information on cluster analysis can be
  found in \cite{ClusterAnalysis}.

  \subfile{tex/hierarchical_clustering}
  \subfile{tex/hier_algo_analysis}
  \subfile{tex/retrieval_mechanics}

\part{Optimization of wavelet density estimation}
  \subfile{tex/wavelet_optimization}

\part{Shape L'Ane Rouge: sliding wavelets}
  % TODO: write intro
  \subfile{tex/sliding_wavelets}
  \subfile{tex/linear_assignment}

\newpage
\bibliographystyle{plain}
\bibliography{tech_report_1.bib}

\end{document}
