\documentclass[../tech_report_1.tex]{subfiles}
\graphicspath{{img}{../img/}}
\begin{document}


\part*{K-means clustering}

$k$-means clustering is a simple and popular algorithm for the \textit{clustering
problem}, the task of grouping a set of observations so that a group
is ``similar'' within itself and ``dissimilar'' to other groups.
$k$-means partitions $n$ observations into $k$ clusters, with each
observation belonging to the nearest mean of the cluster. This problem
is NP-hard in general, but there are heuristics which guarantee convergence
to a local optimum.

The standard heuristic (known as \textit{Lloyd's algorithm}) is the
following:

\begin{algorithm}
\caption{Lloyd's algorithm for k-means clustering}


\begin{algorithmic}[1]
\State{generate an initial set of $k$ means}
\While{not converged}
\State{assign all data points to nearest Euclidean-distance mean}
\State{calculate new means to as the centroids of the observations in the cluster}
\EndWhile
\end{algorithmic} 
\end{algorithm}


There is a choice of initialization method. The \textit{Forgy method}
randomly picks $k$ observations as initial means, while the \textit{Random
Partition method} randomly picks a cluster for each observation.

The Lloyd's algorithm is a heuristic, so it does not guarantee a global
optimum. Furthermore, there exists sets of points in which it converges
in exponential time. However, it has been shown to have a smoothed
polynomial running time, and in practice converges quickly.

\bibliography{../tech_report_1.bib}
\bibliographystyle{plain}
\end{document}
