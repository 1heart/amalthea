\documentclass[../tech_report_1.tex]{subfiles}
\graphicspath{{img/}{../img/}}
\begin{document}

\section*{Spherical k-means clustering}

Spherical k-means clustering the same idea, but with points on a sphere.
We investigated a MATLAB implementation by Nguyen\cite{nguyen2008gene,nguyen_spherical_clustering},
which required a mean-and-norm-normalized dataset located on a hypersphere.
Important aspects of this implementation include:
\begin{itemize}
\item When there exists an empty cluster, the largest cluster is split
\item Use the dot product as ``negative distance'', which leverages the
fact that observations are unit vectors on the hypersphere
\item Use the normalized sum of observations as a centroid/mean, which leverages
the fact that observations are unit vectors on the hypersphere. Note
that this fails on pathological cases where the sum of observations
is zero.
\end{itemize}

\end{document}
